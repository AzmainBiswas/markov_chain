\chapter{Introduction}
The Weak Law of Large Numbers(WLLN) and the Central Limit Theorem(CLT) were proved around  1880's. At that time those were the focal probabilistic issues.
In 1902 Russian mathematician and philosopher P.A. Nekrasov claimed that pairwise independent of random variables were necessary condition for CTL to hold. 
This motivated Andrei A. Markov to construct a counterexample to disprove the claim of P.A. Nekrasov. So he developed Markov Chain in 1907 to disprove a 
"Mathematical proof of freewill" which had been proposed by P.A.Nekrasov to defend the church.

 A Markov chain is a stochastic model that describes a sequence of possible
events or transitions from one state to another of a system. The probability of transitions
from state to state only depends on the current state of the system. A Markov Chain is
used to unravel predictions about future states of a stochastic process using only knowledge
of the present state. This property of “forgetting” past states is known as the memoryless
property. This memoryless property allows for efficient and concise modeling of systems with a large number of states and complex transition dynamics.

Markov chains provide a wide range of analytical methods and tools for comprehending how systems behave over time. The long-term behaviour and stability of a Markov chain can be ascertained by computing transition probabilities and studying steady-state distributions. Additionally, ideas like recurrent and transient states, absorbing states, and mean first passage time shed light on a number of the underlying system's properties and characteristics.

Markov chains have a wide range of uses in many different fields. They have been critical in the modelling and analysis of numerous real-world phenomena, including financial markets, biological processes, social networks, and line-waiting systems. Additionally, Markov chains have found application in decision-making algorithms, search engines, data analysis, and speech recognition.

This paper aims to provide a comprehensive introduction to Markov chains, covering their fundamental concepts, properties, and applications. 

In the upcoming section, we will look at the definition and fundamental ideas of Markov chains, as well as look at their characteristics, analytical methods, and a variety of uses. Readers will have a strong foundation by the end to further investigate.
