\chapter{Introduction}
The Weak law of large number(WLLN) and the Central Limit theorem(CTL) was proved around  1880's. At that time those were the focal probabilistic issues.
In 1902 Russian Mathematician cum Philosopher P.A. Nekrasov claimed that pairwise independent of random variables were necessary condition for CTL to hold. 
This motivated Andrei A. Markov to construct a counterexample to disprove the claim of P.A. Nekrasov. So He developed Markov Chain in 1907 to disprove a 
"Mathematical proof of freewill" which had been proposed by P.A.Nekrasov to defend the church.

 A Markov chain is a stochastic model that describes a sequence of possible
events or transitions from one state to another of a system. The probability of transitions
from state to state only depends on the current state of the system. A Markov Chain is
used to unravel predictions about future states of a stochastic process using only knowledge
of the present state. This property of “forgetting” past states is known as the memoryless
property.

Now a days Markov Chain is a very important concept, it is used in various fields like computer science, mathematics, physics, economics, and more.
Markov chains have proven to be an effective modelling and analysis method for systems that change over time.

